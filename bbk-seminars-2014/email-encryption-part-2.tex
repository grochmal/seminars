\documentclass[hyperref={colorlinks=true}]{beamer}

\usepackage[utf8]{inputenc}
\usepackage[T1]{fontenc}
\usepackage{lmodern}
\usepackage{textcomp}

%\usepackage{multirow}

\mode<presentation>
{
  \usetheme{Warsaw}
  \usecolortheme{seahorse}
  \usecolortheme{orchid}
  \setbeamercovered{transparent}
}

\title[PERSONAL CRYPTOGRAPHY - EMAIL ENCRYPTION 2]{PERSONAL CRYPTOGRAPHY \\
                                                   EMAIL ENCRYPTION \\
                                                   Part 2: communicating}
\author{Michal Grochmal
  <\href{mailto:grochmal@member.fsf.org}{grochmal@member.fsf.org}>}
\institute{Queen Mary University of London}
\date{24th of March 2014}
\subject{Information Security}

\begin{document}

\begin{frame}
  \titlepage
\end{frame}

\begin{frame}{Outline}
  \tableofcontents[pausesections]
\end{frame}

\section{What GnuPG does}

\subsection{Encrypting files}
\begin{frame}[fragile]\frametitle{Encrypting and signing files}
  \begin{itemize}
    \item GnuPG is a tool to encrypt streams of bytes, not only emails.
    \item Understanding how GnuPG encrypts, decrypts and signs any stream of
bytes will help us understand what it actually does to emails.
    \item You can encrypt a file using asymmetric or symmetric cryptography.
    \item GnuPG has a \verb|--verbose| flag, or just \verb|-v| for short.
Using \verb|-vv| shows what GnuPG is doing in detail, and \verb|-vvv| outputs
debug information.  If you are in doubt about what GnuPG is doing use
\verb|gpg -vvv ...|.
  \end{itemize}
\end{frame}

\begin{frame}[fragile]\frametitle{Encrypt using an asymmetric cypher}
  \begin{itemize}
    \item Actually, a hybrid cypher is used.
  \end{itemize}
{\footnotesize
\begin{verbatim}
[~]$ echo "super secret stuff" >> msg
[~]$ gpg -vv -a -e -r grochmal@example.org -o msg.asc msg
gpg: using PGP trust model
gpg: key C870C4F6: accepted as trusted key
gpg: key 0EBE0D44: accepted as trusted key
gpg: using subkey 91BA249B instead of primary key C870C4F6
gpg: This key belongs to us
gpg: reading from `msg'
gpg: writing to `msg.asc'
gpg: RSA/AES256 encrypted for: "91BA249B Michal Grochmal <...>"
\end{verbatim}
}
\end{frame}

\begin{frame}[fragile]\frametitle{Encrypt using a symmetric cypher}
{\footnotesize
\begin{verbatim}
[~]$ gpg -a -c -o msg-sym.asc msg
[~]$ cat msg-sym.asc
-----BEGIN PGP MESSAGE-----
Version: GnuPG v2.0.22 (GNU/Linux)

jA0EAwMC5V2uXllXmQK3ySzqw8HwJ7V4478211bn9C16nyflh1T/lGT/jTwLEcN7
xiIw4GndX0z3CayQXA==
=DMot
-----END PGP MESSAGE-----
[~]$ cat msg.asc
-----BEGIN PGP MESSAGE-----
Version: GnuPG v2.0.22 (GNU/Linux)

hQEMA8gpV8eQuiSbAQf+KyCukr8MQdDAFvPE3y9HSY0oDhbxodCNmkNK/4CjgfnC
...
x9IYzVAAS8u7aCDurWgdhrXw
=qhIT
-----END PGP MESSAGE-----
\end{verbatim}
}
\end{frame}

\begin{frame}[fragile]\frametitle{Decrypting}
{\footnotesize
\begin{verbatim}
[~]$ gpg -d msg.asc
You need a passphrase to unlock the secret key for
user: "Michal Grochmal <grochmal@example.org>"
2048-bit RSA key, ID 91BA249B, created 2014-08-21 (...)

gpg: encrypted with 2048-bit RSA key, ID 91BA249B, ...
      "Michal Grochmal <grochmal@example.org>"
super secret stuff
[~]$ gpg -d msg-sym.asc
gpg: CAST5 encrypted data
gpg: encrypted with 1 passphrase
super secret stuff
gpg: WARNING: message was not integrity protected
\end{verbatim}
}
  \begin{itemize}
    \item The command is the same, how gpg knows how to decrypt the data?
  \end{itemize}
\end{frame}

\begin{frame}[fragile]\frametitle{A GnuPG message}
  \begin{itemize}
    \item Actually an openPGP message, GnuPG only adds the ASCII armour.
  \end{itemize}
{\scriptsize
\begin{verbatim}
[~]$ gpg -vv -d msg.asc
gpg: armor: BEGIN PGP MESSAGE
Version: GnuPG v2.0.22 (GNU/Linux)
:pubkey enc packet: version 3, algo 1, keyid C82957C791BA249B
        data: [2046 bits]
gpg: armor header:
gpg: public key is 91BA249B
gpg: using subkey 91BA249B instead of primary key C870C4F6
...
gpg: public key encrypted data: good DEK
:encrypted data packet: length: 81 mdc_method: 2
gpg: encrypted with 2048-bit RSA key, ID 91BA249B, ...
gpg: AES256 encrypted data
:compressed packet: algo=2 :literal data packet: mode b (62), ...
gpg: original file name='msg'
super secret stuff
gpg: decryption okay
\end{verbatim}
}
\end{frame}

\subsection{Signing}
\begin{frame}[fragile]\frametitle{Sign and compress the message}
{\footnotesize
\begin{verbatim}
[~]$ gpg -a -o msg-sign.asc -s msg
You need a passphrase to unlock the secret key for
user: "Michal Grochmal <grochmal@example.org>"
2048-bit RSA key, ID 8A0ACBDA, created 2015-02-15 (...)

[~]$ cat msg-sign.asc
-----BEGIN PGP MESSAGE-----
Version: GnuPG v2.0.22 (GNU/Linux)

owEBTAGz/pANAwACAXFPKDF6CsvaAawcYgNtc2dVC0/cc3VwZXIgc2VjcmV0IHN0
...
=SriB
-----END PGP MESSAGE-----

[~]$ gpg --verify msg-sign.asc
gpg: Signature made Thu 19 Mar 2015 10:38:20 PM UTC ...
gpg: Good signature from "Michal Grochmal <...>"
\end{verbatim}
}
\end{frame}

\begin{frame}[fragile]\frametitle{Sign a clear message}
{\scriptsize
\begin{verbatim}
[~]$ gpg -a -o msg-sign.asc --clearsign msg
You need a passphrase to unlock the secret key for
user: "Michal Grochmal <grochmal@example.org>"
2048-bit RSA key, ID 8A0ACBDA, created 2015-02-15 (...)
[~]$ cat msg-sign.asc
-----BEGIN PGP SIGNED MESSAGE-----
Hash: SHA1

super secret stuff
-----BEGIN PGP SIGNATURE-----
Version: GnuPG v2.0.22 (GNU/Linux)

iQEcBAEBAgAGBQJVC1GhAAoJEHFPKDF6CsvayoIH/RXj+DNcGxEmMOP+4GFJuGiD
...
=Hjiw
-----END PGP SIGNATURE-----
[~]$ gpg --verify msg-sign.asc
gpg: Signature made Thu 19 Mar 2015 10:45:53 PM UTC ...
gpg: Good signature from "Michal Grochmal <...>"
\end{verbatim}
}
\end{frame}

\begin{frame}[fragile]\frametitle{Detached signature}
{\footnotesize
\begin{verbatim}
[~]$ gpg -a -o msg-sign.asc -b msg
You need a passphrase to unlock the secret key for
user: "Michal Grochmal <grochmal@example.org>"
2048-bit RSA key, ID 8A0ACBDA, created 2015-02-15 (...)
[~]$ cat msg-sign.asc
-----BEGIN PGP SIGNATURE-----
Version: GnuPG v2.0.22 (GNU/Linux)

iQEcBAABAgAGBQJVC1L1AAoJEHFPKDF6CsvabIUH/j6Jdwatc8MpO87XqDxm4SOs
...
=GPnh
-----END PGP SIGNATURE-----
[~]$ gpg --verify msg-sign.asc  # Ops!
gpg: no signed data
gpg: can't hash datafile: No data
[~]$ gpg --verify msg-sign.asc msg
gpg: Signature made Thu 19 Mar 2015 10:51:33 PM UTC ...
gpg: Good signature from "Michal Grochmal <...>"
\end{verbatim}
}
\end{frame}

\begin{frame}[fragile]\frametitle{Combine with encryption}
{\scriptsize
\begin{verbatim}
[~]$ gpg -a -s -e -r grochmal@example.org -o msg-sign-enc.asc msg
You need a passphrase to unlock the secret key for
user: "Michal Grochmal <grochmal@example.org>"
2048-bit RSA key, ID 8A0ACBDA, created 2015-02-15 (...)
[~]$ cat msg-sign-enc.asc
-----BEGIN PGP MESSAGE-----
Version: GnuPG v2.0.22 (GNU/Linux)

hQEMA8gpV8eQuiSbAQf+OXnwvgroSJi9/cGJLfe+0KpYfIX0i4t/36Lqu7Lpe7cK
...
-----END PGP MESSAGE-----
[~]$ gpg -d msg-sign-enc.asc
You need a passphrase to unlock the secret key for
...
gpg: encrypted with 2048-bit RSA key, ID 91BA249B, ...
      "Michal Grochmal <grochmal@example.org>"
super secret stuff
gpg: Signature made Thu 19 Mar 2015 11:06:36 PM UTC ...
gpg: Good signature from "Michal Grochmal <...>"
\end{verbatim}
}
\end{frame}

\subsection{Messages}
\begin{frame}[fragile]\frametitle{Attachments, encryption and singing}
  \begin{itemize}
    \item Shall the signature be inside or outside the encryption?  Depends on
the media through which the message will go through.
    \item Now that we can encrypt and sign different types of files (not only
emails) we can use encrypted attachments.
    \item Note that binary files can be encrypted in the same fashion:
{\footnotesize
\begin{verbatim}
[~]$ gpg -a -s -e -r grochmal@example.org -o l.asc l.png
[~]$ gpg -o l2.png -d l.asc
[~]$ diff l.png l2.png
[~]$
\end{verbatim}
}
  \end{itemize}
\end{frame}

\section{Communicate with others}

\subsection{Signed keys and Trust}
\begin{frame}{Web of Trust}
  \begin{itemize}
    \item We know what GnuPG does with its encryption and singing functions.
And that it is not extremely complex.  Yet we were using only a single key in
all examples above.
    \item Key management is the weakest point of cryptography.
    \item The \emph{keychain} maintained by GnuPG is a great part of the
functionality of the program.
    \item Your private keys, all public keys you need to communicate with other
people and several certificates (we will talk about these later) are in the
keychain.
    \item You exchange keys with other people and your keychain grows, yet
exchanging keys with every person you want to communicate is likely impossible.
    \item Thus the \emph{web of trust} is a way of trusting a key of a person
that you have never met, yet you have common friends.
  \end{itemize}
\end{frame}

\begin{frame}{Signing keys}
  \begin{itemize}
    \item The web of trust starts with the feature that you can sign keys of
other people.  The fact that you sign someone's key means that you are certain
that that key belongs to that person.
    \item You shall really be certain that someone is the person he claims to
be before signing their key.
    \item When Bob signs Alice's key he is using his private key to encrypt
Alice's public key, then Bob sends to Alice the cyphertext.  Anyone who has a
copy of Bob's public key can get Alice's key from this cyphertext and compare
it with their copy of Alice's public key.
    \item There is no reason to sign a private key.  It should be secret!
    \item Signatures of keys remain on the keychain.
    \item It can be said that people collect signatures on their keys, with
more signatures it is easier that someone can validate their key without
personally checking it.
  \end{itemize}
\end{frame}

\begin{frame}{Trust in a key signature}
  \begin{itemize}
    \item The signature model would work well if everyone signed other people's
keys extremely carefully.  Yet, you cannot assume that that is the case!
    \item \emph{Trust} is \emph{your} evaluation of how well someone checks
keys before signing them.
    \item Trust is a completely private measure: whilst you send signatures of
their keys to their owners, you never need to reveal your trust in someone.
    \item While key signatures are a measure of how many people can assure that
you are who you claim to be, trust is your own personal measure of how much you
trust someone you know to assure you of the identity of a third person.
    \item Remember this!  Signatures and trust are very different things!
  \end{itemize}
\end{frame}

\begin{frame}{Signing keys vs. Trusting people}
  \begin{itemize}
    \item Bob and Alice met and signed their keys.
    \item Alice and Charlie met and signed their keys.
    \item If Bob \emph{fully} trusts that Alice do not sign keys without
checking them properly he can check Charlie's key using Alice's signature on
that key.  Then Bob can communicate with Charlie.
    \item This does not mean that Charlie can communicate with Bob!  That
depends on Charlie's trust in Alice's carefulness.
  \end{itemize}
\end{frame}

\begin{frame}[fragile]\frametitle{Levels of Trust}
  \begin{itemize}
    \item GnuPG allows different trust levels: "don't know" (the default), "do
not trust", "trust marginally", "trust fully" and "trust ultimately".
    \item "don't know" and "do not trust" are self explanatory.
    \item "ultimate trust" is for your own keys.
    \item Finally, "full" and "marginal" trust irradiates to the keys that are
signed by the trusted people.  To consider a key valid (by default) it must be
signed by yourself, by a person you fully trust or by three people which you
marginally trust.
    \item \verb|--completes-needed| and \verb|--marginals-needed| can be used
with \verb|gpg| to tune this number.
    \item Note that their trust in other people has nothing to do with your
trust.
    \item Note that you can trust people you have never met and signed a key,
it is your choice!
  \end{itemize}
\end{frame}

\begin{frame}[fragile]\frametitle{Sign a key with GnuPG}
  \begin{itemize}
    \item Remember that \verb|gpg --edit-key <key>| has a menu of subcommands.
    \item Use \verb|gpg --fingerprint <key>| or \verb|fpr| subcommand and check
the fingerprint with the owner of the key.
    \item Use \verb|gpg --sign-key <key>| or \verb|sign| subcommand to sign it
with your key.
    \item Use \verb|gpg --check-sigs <key>| or \verb|check| subcommand to
verify if a signature matches.
    \item If public keys are missing the signatures will not be verified, use
\verb|gpg --list-sigs| for a list of the keys used in the signatures.
    \item If you're using the subcommands in \verb|gpg --edit-key <key>| you
need to \verb|save| the key after signing it.
  \end{itemize}
\end{frame}

\begin{frame}[fragile]\frametitle{Sign a key, command line example}
{\scriptsize
\begin{verbatim}
[~]$ gpg --edit-key 0x76D52AC6
pub  1024D/76D52AC6  created: 2003-06-03  expires: 2018-02-03  usage: S
                     trust: unknown       validity: unknown
sub  1024g/1206F606  created: 2003-06-03  expired: 2013-05-31  usage: E
[ unknown] (1). Michal Grochmal <grochmal@example.com>
gpg> fpr
pub   1024D/76D52AC6 2003-06-03 Michal Grochmal <grochmal@example.com>
 Primary key fingerprint: 4673 48C8 5F3A 5A9B 2B0D  ...
gpg> sign
Really sign all user IDs? (y/N) y
...
Really sign? (y/N) y
You need a passphrase to unlock the secret key for$
user: "Michal Grochmal (hacker) <grochmal@example.org>"$
2048-bit RSA key, ID B713DB95, created 2015-02-21$
gpg> check
uid  Michal Grochmal <grochmal@example.com>
sig!3  C840C4F6 2015-02-19  [self-signature]
sig!   B713DB95 2015-02-21  Michal Grochmal (hacker) <...>
gpg> save
\end{verbatim}
}
\end{frame}

\begin{frame}[fragile]\frametitle{Trust a person in GnuPG}
{\scriptsize
\begin{verbatim}
[~]$ gpg --edit-key 0x76D52AC6
pub  1024D/76D52AC6  created: 2003-06-03  expires: 2018-02-03  usage: S
                     trust: unknown       validity: unknown
sub  1024g/1206F606  created: 2003-06-03  expired: 2013-05-31  usage: E
[ unknown] (1). Michal Grochmal <grochmal@example.com>

gpg> trust
Please decide how far you trust this user
to correctly verify other users' keys
(by looking at passports, checking
fingerprints from different sources, etc.)
  1 = I don't know or won't say
  2 = I do NOT trust
  3 = I trust marginally
  4 = I trust fully
  5 = I trust ultimately
  m = back to the main menu
Your decision? 4

gpg> save
\end{verbatim}
}
\end{frame}

\subsection{Keyservers}
\begin{frame}{Keyservers}
  \begin{itemize}
    \item Exchanging keys in person or through email is painful.  Keyservers
were created to automate key exchange.
    \item You upload your key to a keyserver and others can download it, you
also can download other people's keys from the keyserver.  When you sign
someone's key you can upload it to the keyserver as well.
    \item Keyservers can only add data to a key, never delete anything.  This
is because the entire web of trust is not in sync all the time.
    \item Keyservers sync keys between themselves.
    \item Be careful when uploading a non-expiring key to a keyserver, it will
stay there forever.  It will be bound to your name forever.
    \item How can we manage to delete keys or UIDs that get compromised?  The
answer is certificates.
  \end{itemize}
\end{frame}

\begin{frame}[fragile]\frametitle{Using a keyserver with GnuPG}
{\footnotesize
\begin{verbatim}
[~]$ gpg --keyserver pgp.mit.edu --recv-key 0xC840C4F6
gpg: requesting key C839C4F6 from hkp server pgp.mit.edu
gpg: key C840C4F6: public key "Michal Grochmal <...>"
imported
gpg: Total number processed: 1
gpg:               imported: 1  (RSA: 1)

[~]$ gpg --keyserver pgp.mit.edu --send-key 0xC840C4F6
gpg: sending key C840C4F6 to hkp server pgp.mit.edu
[~]$
\end{verbatim}
}
  \begin{description}
    \item[Some public keyservers are] \hfill \\
hkp://pgp.mit.edu, hpk://keys.gnupg.net, http://http-keys.gnupg.net,
mailto:pgp-public-keys@keys.nl.pgp.net, hpk://wwwkeys.uk.pgp.net
  \end{description}
\end{frame}

\begin{frame}[fragile]
\frametitle{Certificates, what the heck these actually are}
  \begin{itemize}
    \item In simple terms, a certificate is a public key signed with the
private key.  Integrity of the public key is this way achieved.  Then, several
pieces of information can be bound to the public key by signing them with the
private key.
    \item In the case of GnuPG the piece of information bound to the public key
is the UID (your name and email address).
    \item A certificate can contain other information, e.g. signatures with
other private keys.
    \item A certificate can only be created or modified with the possession of
the private key, therefore if the public key is verified the rest of the
information in the certificate can be trusted.
    \item The revocation certificate is a good example.
    \item Your keys and UIDs are actually stored as certificates, see the text
\verb|[self-signature]| in the examples above.
  \end{itemize}
\end{frame}

\section{Consequences and Summary}

\subsection{How all this works with Email}
\begin{frame}{Back to Email}
  \begin{itemize}
    \item Email is just plain text.  When email goes through SMTP it look
pretty much like HTTP: a bunch of headers and then the body.
    \item Email clients actually connect to email servers in the same way that
browsers connect to websites through HTTP.
    \item Email is older than HTTP, it has a concept of email relays that coped
with connectivity issues before the internet (anyone remembers UUCP?).
    \item Unfortunately you cannot easily mix HTTPS and GnuPG encryption,
therefore using a webmail client with GnuPG is not practical.  Your private key
would need to be on the server, which is a bad idea.
    \item If there's still time, talk about bad randoms.
  \end{itemize}
\end{frame}

\subsection{References and further reading}
\begin{frame}{References and further reading}
  \begin{description}
    \item[GNU Privacy Handbook] \hfill \\
\href{https://www.gnupg.org/gph/en/manual.html}
{https://www.gnupg.org/gph/en/manual.html}
    \item[Justin Miller's GnuPG with Mutt Guide] \hfill \\
\href{http://codesorcery.net/old/mutt/mutt-gnupg-howto}
{http://codesorcery.net/old/mutt/mutt-gnupg-howto}
  \end{description}
\end{frame}

\end{document}

